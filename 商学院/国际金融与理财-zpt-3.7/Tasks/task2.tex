\documentclass[UTF8]{ctexart}
\usepackage{dirtree}
\usepackage{listings}
\usepackage{xcolor}
\usepackage{graphicx}
\usepackage{enumerate}
\usepackage[a4paper]{geometry} 
\usepackage{amsmath,amsthm,mathtools}
\usepackage{mathtools}
\usepackage{diagbox}
\usepackage{multirow,makecell}
\usepackage{float}
\usepackage{url}
\usepackage[nottoc]{tocbibind}
\usepackage{float}
\newcommand{\refe}[1]{Eq.\ref{#1}}
\newcommand{\reft}[1]{Theory.\ref{#1}\ }
\newcommand{\reff}[1]{图\ref{#1}\ }
\newtheorem{theorem}{Theory}[section]
\geometry{bottom=2cm,left=1cm,right=1cm}

\title{个人作业二}
\author{张配天-2018202180}
\begin{document}
    \maketitle
    \section{假定加元的即期汇率为1加元=0.85美元,加拿大和美国的通货膨胀率是相同的。预计加拿大将经历4\%的通货膨胀,美国将经历3\%的通货膨胀。根据购买力平价理论,加元的币值将如何变动?变化幅度是多少?变化后的即期汇率为多少?}
    
    \begin{enumerate}[I.]
        \item 加币会贬值
        \item 变化幅度为0.1\%
        \item 变化后的即期汇率为1加元=0.8495美元;\par 首先,加币为外币,美元为本币,有 $$\Delta S_{B\rightarrow A} = \frac{S_{B\rightarrow A}'}{S_{B\rightarrow A}} - 1 \approx l_A - l_B$$\par 代入得\begin{equation}
            S_{B\rightarrow A}' = (1-1\%)*0.85 = 0.8415\$
        \end{equation}
    \end{enumerate}
    
    \section{假定澳大利亚元的即期汇率是0.09美元,而澳大利亚与美国一年期的利率开始为6\%,然后澳大利亚一年期的利率上涨5\%,而美国一年期的利率维持不变。根据国际费雪效应理论,预测一年后的即期汇率}
    \begin{enumerate}[I.]
        \item 一年后澳大利亚元的即期汇率是1澳大利亚元=0.0855美元;\par 首先美元为本币,澳元为外币,根据$$\Delta S_{B\rightarrow A} = \frac{S_{B\rightarrow A}'}{S_{B\rightarrow A}} - 1 \approx \alpha_A - \alpha_B$$\par 代入得\begin{equation}
            S_{B\rightarrow A}' = (1-(6\%*5\%))*0.09 = 0.08973\$
        \end{equation}
    \end{enumerate}

    \section{假定澳大利亚元的即期汇率是0.09美元,而澳大利亚与美国一年期的利率开始为6\%,然后澳大利亚一年期的利率上涨5\%,而美国一年期的利率维持不变。根据利率平价理论,预测一年后的即期汇率。}
    \begin{enumerate}[I.]
        \item 一年后的即期汇率为1澳大利亚元=0.0855美元;\par 首先,美元为本币,澳大利亚元为外币,根据$$\omega = (1+r^B)(1+\Delta k^{B\rightarrow A}) - 1$$
        $$\Delta k^{B\rightarrow A} = \frac{i_f - i_{c}}{i_c} = \frac{1+r^A}{1+r^B} \approx r^A - r^B$$\par 代入得
        \begin{align}
            i_f = (1+6\%*(1-1-5\%))*0.09 = 0.08973\$
        \end{align}
    \end{enumerate}

    \section{假定今天存在下列即期汇率:即期汇率1英镑=1.60美元,180天远期汇率1英镑=1.56美元,180天英国利率为4\%,180天美国利率为3\%。根据这些信息,利率平价存在吗?抛补套利可行吗?如何操作?}
    \begin{enumerate}[I.]
        \item 利率平价不存在;\par 首先,英镑为外币,美元为本币,根据$$\omega = (1+r^B)(1+\Delta k^{B\rightarrow A}) - 1$$
        $$\Delta k^{B\rightarrow A} = \frac{i_f - i_{c}}{i_c} = \frac{1+r^A}{1+r^B} \approx r^A - r^B$$\par 代入得
        \begin{align}
            \Delta k^{B\rightarrow A} = \frac{1.56-1.60}{1.60} = -2.5\%\\
            r^A - r^B = -1\% \neq -2.5\%
        \end{align}
        \item 抛补套利可行;
        \item 签订卖出英镑的远期合约,然后将美元兑换为英镑,在英国投资,到期后再履行合约将其换回美元;
    \end{enumerate}
    \section{课堂案例}
    \begin{enumerate}[I.]
        \item 根据$$\omega = (1+r^B)(1+\Delta S^{B\rightarrow A}) - 1$$ 代入得
        \begin{align}
            \omega^{max} = (1 + 14\%)*(1+40\%) - 1 &= 59.6\%\\
            \omega^{min} = (1 + 14\%)*(1-40\%) - 1 &= -31.6\%
        \end{align}
        如果假设该国家货币实际价值涨/跌的概率都是$50\%$,那么可以计算期望回报率$E(\omega) = 14\% > 0$,因此我还是愿意投资的。
        \item 首先将人民币兑换为该国家货币,然后在该国家买入一年期债券进行投资,到期后履行合约,将该国货币根据合约价$0.39¥$兑换为人民币,根据$$\omega = (1+r^B)(1+\Delta k^{B\rightarrow A}) - 1$$ 代入得\begin{align}
            \omega = (1+14\%)*\frac{0.39}{0.4}-1 = 11.15\%
        \end{align}
        \item 涉及的风险主要在于该国货币一年后实际价值上涨,则相对来说会损失收益。
        \item 进行抛补套利,因为其收益率更高。
    \end{enumerate}
    \end{document}