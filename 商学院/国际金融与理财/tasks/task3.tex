\documentclass[UTF8]{ctexart}
\usepackage{dirtree}
\usepackage{listings}
\usepackage{xcolor}
\usepackage{graphicx}
\usepackage{enumerate}
\usepackage[a4paper]{geometry} 
\usepackage{amsmath,amsthm,mathtools}
\usepackage{mathtools}
\usepackage{diagbox}
\usepackage{multirow,makecell}
\usepackage{float}
\usepackage{url}
\usepackage[nottoc]{tocbibind}
\usepackage{float}
\newcommand{\refe}[1]{Eq.\ref{#1}}
\newcommand{\reft}[1]{Theory.\ref{#1}\ }
\newcommand{\reff}[1]{图\ref{#1}\ }
\newtheorem{theorem}{Theory}[section]
\geometry{bottom=2cm,left=1cm,right=1cm}

\title{个人作业三}
\author{张配天-2018202180}
\begin{document}
\maketitle
% \section{某公司是一家美国公司,1年后公司需要3 000 00 0澳大利亚元支付货款。美元年利率为7\%,而澳大利亚元的年利率为12\%,澳大利亚即期汇率为1澳元=0.85美元,一年远期汇率为1澳元=0.81美元。}
% \subsection{如果公司用货币市场套期保值,如何操作?}
% 借入美元,用即期汇率兑换为澳大利亚元,之后在澳大利亚元市场进行投资,1年后用得到的澳大利亚元支付。
% \subsection{该公司现在需要借入多少美元?1年后偿还贷款所需的美元金额是多少?}
% 设借入$x$美元,则有

% \begin{align}
%     \frac{x}{0.85}\cdot (1+12\%) &= 3000000\\
%     \omega = x(1+7\%)
% \end{align}
% \par 解得$w\approx2436160.7$
% \subsection{该公司用货币市场套期保值好还是用远期合约套期保值好?}
% \begin{equation*}
%     \omega' = 3000000*0.81 = 2430000
% \end{equation*}
% \par 远期合约套期保值
% \section{某公司三个月后会收到2 000 000瑞士法郎。该公司认为三个月的远期汇率就是未来即期汇率的准确预测。三个月瑞士法郎的远期汇率为1瑞士法郎=0.68美元。现在单位瑞士法郎卖出期权的行权价为1瑞士法郎= 0.69美元,期权费1瑞士法郎= 0.03美元。通过计算说明该公司应选择卖出期权套期保值还是不套期保值?}
% \subsection{不套期保值}
% \begin{equation*}
%     \omega_1 = 2000000*0.68 = 1360000
% \end{equation*}
% \subsection{期权套期保值}
% \begin{equation*}
%     \omega_2 = 2000000*(0.69-0.03) = 1320000
% \end{equation*}
% \par 不套期保值
\section{吉姆是一家美国体育出口公司的业主,因出口产品约在一个月后会收到约10 000英镑。吉姆很关心汇率风险,因为他预计有两种可能性:(1)下月英镑会贬值3\%,概率为70\%;(2)下月英镑会升值2\%,概率为30\%。吉姆注意到当前英镑的即期汇率为1.65美元,一个月远期汇率约为1英镑=1.645美元。同时,吉姆可在证券公司的场外市场买入卖出期权,其行权价为1英镑=1.645美元,期权费为1英镑=0.025美元,一个月后到期。}
\subsection{若该公司对一个月后收到的应收账款不进行套期保值,请计算两种汇率情形下收到的美元金额。}
使用即期汇率兑换英镑为美元,两种情况下分别有
    \begin{equation*}
        \omega_1 = 10000 * 1.65(1-0.03) = 15956.5
    \end{equation*}
    \begin{equation*}
        \omega_2 = 10000 * 1.65(1 + 0.02) = 16779
    \end{equation*}
    \par 期望有\begin{equation*}
        \omega_3 = 0.7 * \omega_1 + 0.3 * \omega_2 = 16203.25
    \end{equation*}
\subsection{若该公司用卖出期权对一个月后收到的应收账款套期保值,请计算两种情形下收到的美元金额}
如果对应即期汇率低于行权价,则行权;否则不行权,两种情况下分别有
    \begin{equation*}
        \omega_4 = 10000 *(1.645-0.025) = 16200
    \end{equation*}
    \begin{equation*}
        \omega_5 = 10000 * (1.645* (1 + 0.02) - 0.025) = 16529
    \end{equation*}
    \par 期望有
    \begin{equation*}
        \omega_6 = 0.7 * \omega_4 + 0.3 * \omega_5 = 16289.7
    \end{equation*}
\subsection{如果用远期交易,结果又如何}
使用远期汇率卖出英镑,两种情况下分别有    
\begin{equation*}
        \omega_7 = 10000 *1.645 = 16450
    \end{equation*}
\subsection{公司应该如何操作}
由于$\omega_7 > \omega_6 > \omega_3$,因此使用远期汇率套期保值。

\section{某美国公司180天后需要200 000英镑,现考虑用以下四种方式筹集资金:(1)远期套期保值(2)货币市场套期报值(3)期权套期保值(4)不套期保值。该公司的分析师取得以下信息,可用于评价不同方案。(1)英镑的即期汇率1英镑=1.50美元;(2)180天英镑的远期汇率为1英镑=1.47美元;(3)180天英镑买入期权的行权价为1英镑=1.48美元,期权费为1英镑=0.03美元;(4)180天英镑卖出期权费的行权价为1英镑=1.49美元,期权费为1英镑=0.02美元;(5)预计180天后的即期汇率、利率如表4所示。请说明该公司应选择何种方案?}
\begin{enumerate}
    \item \textbf{远期套期保值:},使用远期汇率兑换相应英镑,应支出
    \begin{equation*}
        \omega_1 = 200000 * 1.47 = 294000
    \end{equation*}
    \item \textbf{货币市场套期保值:}借入$x$美元,用即期汇率兑换为英镑,之后在英镑市场进行投资,180天后用得到的英镑支付账款,且将当前借入的美元
    \begin{align*}
        \frac{x}{1.5} \cdot (1+4.5\%) &\approx 287081.34\\
        \omega_2 = x*(1+5\%) &\approx 301435.41
    \end{align*}
    \item \textbf{期权套期保值:}如果即期汇率高于行权价,则行权,否则不行权,每一种情况下有
        \begin{equation*}
            \omega_3 = 20000 * (1.48+0.03) = 302000
        \end{equation*}
         \begin{equation*}
            \omega_4 = 20000 * (1.43+0.03) = 292000
        \end{equation*}
         \begin{equation*}
            \omega_5 = 20000 * (1.46+0.03) = 298000
        \end{equation*}
    \par 因此有期望付款\begin{equation*}
        \omega_6 = 0.2 * \omega_3 + 0.7 * \omega_4 + 0.1 * \omega_5 = 297200
    \end{equation*}
    \item \textbf{不套期保值}
        \begin{equation*}
            \omega_7 = 200000 * (1.43*0.2 + 1.46 * 0.7 +1.52*0.1) = 292000
        \end{equation*}
    
    \par 因为$\omega_7 < \omega_1 < \omega_6 < \omega_2$,因此可以选择不套期保值。    
\end{enumerate}
\section{一家美国的跨国公司在美国和德国从事经营。为了评估经济风险,它汇编了下列资料。
对美国的销售收入在一定程度上受德国马克的影响,因为来自德国出口上的竞争非常激烈。它依据表6的汇率情形预测在美国的销售收入。
对德国的销售以马克计价;预计马克收入为6亿马克。
预计来自美国采购原材料的销售成本为2亿美元,来自德国的为1亿马克。
估计固定经营与管理费用为3 000万美元(不包括利息费用)。估计可变经营与管理费用为销售收入总额(包括德国销售收入折算为美元)的20\%。
估计美元贷款的利息费用为2 000万美元;公司没有马克贷款。
}
\subsection{依据三种可能汇率情形,编制该公司的预测利润表}
\begin{table}[H]
    \centering
    \begin{tabular}[]{lccc}
        \hline
        汇率&0.48美元&0.50美元&0.54美元\\
        \hline
        销售收入(百万美元):&\\
        美国:&100&105&110\\
        德国:&288&300&324\\
        合计:&388&405&434\\
        \hline
        销售成本(百万美元):&\\
        美国:&200&200&200\\
        德国:&48&50&54\\
        合计:&248&250&254\\
        \hline
        销售毛利(百万美元):&140&155&180\\
        \hline
        固定部分及经营及管理费用:(百万美元):&30&30&30\\
        变动部分:&77.6&81&86.8\\
        合计:&107.6&111&116.8\\
        \hline
        税前利润(百万美元):&32.4&44&63.2\\
        \hline
    \end{tabular}
\end{table}
\subsection{通过计算说明公司经营风险的大小}


\end{document}