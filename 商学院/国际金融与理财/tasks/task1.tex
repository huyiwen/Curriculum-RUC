\documentclass[UTF8]{ctexart}
\usepackage{dirtree}
\usepackage{listings}
\usepackage{xcolor}
\usepackage{graphicx}
\usepackage{enumerate}
\usepackage[a4paper]{geometry} 
\usepackage{amsmath,amsthm,mathtools}
\usepackage{mathtools}
\usepackage{diagbox}
\usepackage{multirow,makecell}
\usepackage{float}
\usepackage{url}
\usepackage[nottoc]{tocbibind}
\usepackage{float}
\newcommand{\refe}[1]{Eq.\ref{#1}}
\newcommand{\reft}[1]{Theory.\ref{#1}\ }
\newcommand{\reff}[1]{图\ref{#1}\ }
\newtheorem{theorem}{Theory}[section]
\geometry{bottom=2cm,left=1cm,right=1cm}

\title{个人作业一}
\author{张配天-2018202180}
\begin{document}
    \maketitle
    \section{}
    \subsection{}
    卖出/买入差价为
    \begin{align*}
        %P_b =& \frac{1}{0.784}\\
        %P_s =& \frac{1}{0.8}\\
        \Delta =& \frac{P_s - P_b}{P_b} = \frac{0.8-0.784}{0.784} \approx 2.04\%
    \end{align*}
    \subsection{}
    由于即期利率低于远期利率,因此贴水为
    \begin{equation*}
        \frac{prenium}{discount} = \frac{0.19-0.188}{0.190} * \frac{360}{90} \approx 4.21\% 
    \end{equation*}    
    \section{}
    \subsection{}
    \begin{itemize}
        \item 地点套汇可行;
        \item 在银行乙卖出美元,买入瑞士法郎,之后在银行甲卖出瑞士法郎,买入美元;1瑞士法郎最多对应$0.401-0.400=0.001\$$的套汇利润;
        \item 会有很多人套汇,导致银行乙瑞士法郎需求增多,从而价格升高,银行甲瑞士法郎共计增多,从而价格降低,直至两者平衡;
    \end{itemize}
    \subsection{}
    首先换算成同一标价法:\begin{itemize}
        \item $1 \$ = \frac{1}{0.15} = \frac{20}{3} F $;
        \item $1 f = \frac{1}{4}\ Mark$;
    \end{itemize}
    \begin{equation}
        \mathcal{S} = 0.6*\frac{20}{3}*0.25 =1
    \end{equation}
    \par 因此,没有三角套汇的可能性。
    \section{}
    \begin{itemize}
        \item 因为行权价低于即期汇率,因此\textbf{应该行权;}
        \item 每单位净利润$\omega = 0.65-0.6 = 0.05\$$
        \item 每合同净利润$W = (0.65-0.60)*65200 = 3260\$$
        \item 盈亏平衡的即期利率$r = 0.60 + 0.06 = 0.66\$$
    \end{itemize}
\end{document}