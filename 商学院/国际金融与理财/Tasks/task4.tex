\documentclass[UTF8]{ctexart}
\usepackage{dirtree}
\usepackage{listings}
\usepackage{xcolor}
\usepackage{graphicx}
\usepackage{enumerate}
\usepackage[a4paper]{geometry} 
\usepackage{amsmath,amsthm,mathtools}
\usepackage{mathtools}
\usepackage{diagbox}
\usepackage{multirow,makecell}
\usepackage{float}
\usepackage{url}
\usepackage[nottoc]{tocbibind}
\usepackage{float}
\newcommand{\refe}[1]{Eq.\ref{#1}}
\newcommand{\reft}[1]{Theory.\ref{#1}\ }
\newcommand{\reff}[1]{图\ref{#1}\ }
\newtheorem{theorem}{Theory}[section]
\geometry{bottom=2cm,left=1cm,right=1cm}
\author{张配天-2018202180}
\title{个人作业四}
\begin{document}
    \maketitle
    \section{假定某中国公司考虑发行德国马克债券。目前票面利率是7\%,并且该公司没有德国马克收入用来支付债券本息。该公司主要是看到了德国马克债券融资利率低,因为在中国发行的人民币债券票面利率为12\%。假定每种债券都是4年到期并可以以面值出售。该公司需要借入1 000万元人民币,因此它或者发行面值为1 000万元的人民币债券,或者发行面值为2 000万元的德国马克债券。德国马克即期汇率是0.50元人民币。该公司预计今后4年每年年末的德国马克汇率如下表所示。利息在年末支付。计算德国马克每年的实际融资利率。该公司应该发行何种债券?}
    现金流量表如图:\begin{table}[H]
        \centering
        \begin{tabular}[]{ccccc}
            \hline
            融资方法&1&2&3&4\\
            马克支付额&$2000*7\% = 140$&140&140&2140\\
            换算为人民币&72.8&78.4&81.2&1134.2\\
            人民币债券支付额&120&120&120&1120\\
            \hline
        \end{tabular}
    \end{table}
    计算实际融资利率为\begin{align}
        r_{mark} &= \frac{1}{4} * \frac{72.8+78.4+81.2+134.2}{1000} = 9.165\%\\
        r_{yuan} &= 12\%
    \end{align}
    \par 易得$r_{mark} < r_{yuan}$, 因此发型德国马克债券。
        
\end{document}