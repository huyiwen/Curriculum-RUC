\documentclass[UTF8]{ctexart}
\usepackage{dirtree}
\usepackage{listings}
\usepackage{xcolor}
\usepackage{graphicx}
\usepackage{enumerate}
\usepackage[a4paper]{geometry} 
\usepackage{amsmath}
\usepackage{mathtools}
\usepackage{diagbox}
\usepackage{multirow,makecell}

\title{Lab1-Shell编程}
\author{张配天-2018202180}
\date{\today}
\geometry{bottom=2cm}
\lstset{
 columns=fixed,       
 numbers=left,                                        % 在左侧显示行号
 numberstyle=\tiny\color{gray},                       % 设定行号格式
 basicstyle=\ttfamily,
 frame=none,                                          % 不显示背景边框
 backgroundcolor=\color[RGB]{245,245,244},            % 设定背景颜色
 keywordstyle=\color[RGB]{40,40,255},                 % 设定关键字颜色
 numberstyle=\footnotesize\color{darkgray},           
 commentstyle=\color{gray}\ttfamily,                % 设置代码注释的格式
 stringstyle=\rmfamily\slshape\color[RGB]{128,0,0},   % 设置字符串格式
 showstringspaces=false,
 breaklines=true,
 language=python
}
\begin{document}
    \kaishu
    \maketitle
    \thispagestyle{headings}
    \tableofcontents
    %\pagestyle{headings}
    \newpage
    \section{实验目的}
    编写一个 shell 脚本程序,功能是取出指定目录中的所有 C 语言源程序文件,并对其编译和执行。 
    \section{实验方法}
    注意到ubuntu的特性,在/home/目录下vim的新文件也不是全权限,因此先自己实现了一个小脚本myvi保存在/usr/bin/下,
    用来创造全权限的文件并转至vim编辑\\
    \emph{展示代码如下:}
    \begin{lstlisting}[] 
    fileName="$1"
    echo "FILE:${fileName}"
    touch $fileName
    chmod 777 $fileName
    vim $fileName
    \end{lstlisting} 

    \section{程序清单}
    \dirtree{%
        .1 /. 
        .2 /home. 
        .3 /pt. 
        .4 /shell.
        .5 lab1.sh.  
        .4 /c. 
        .5 /directory.
        .6 5.c. 
        .5 1.c. 
        .5 2.c. 
        .5 3.c. 
        .5 4.c.
        .5 NoneC.txt.
        .4 /桌面.
        .5 outputHeart.c.
        .5 text.txt.
        .5 sad.py.
    }
    \newpage
    \paragraph{其中}
    \begin{itemize}
        \item /桌面 , /shell , /c , /directory均为目录
        \item /shell目录下保存着可执行的Shell脚本lab1.sh
        \item /c目录下保存着4个可编译.c文件,还保存着/directory目录,用来完成对遇到目录的特殊情况处理
        \item /c目录下另有一个非C语言写成的NoneC.txt文件,用来完成对编译错误的特殊情况处理
        \item /directory目录下保存着最后一个可编译的.c文件
        \item /桌面目录下保存着可编译的outputHeart.c和不可编译的text.txt文件,将来可作为第二个参数传入脚本
    \end{itemize}

    \subsection{C程序运行结果}
    %\begin{center}
        1.c输出为“张”\par
        2.c输出为“配”\par
        3.c输出为“天”\par
        4.c输出为“晚安”\par
        5.c输出为由*组成的V字样\par
        outputHeart.c输出心形图案\par
        %\newpage
    %\end{center}
    \subsection{Shell脚本代码}
    \begin{lstlisting}
#定义编译链接的函数,将目录作为参数传递进入函数
compile_run_C(){
    #遍历传入目录下所有文件和目录
    for file in $(ls $1) 
    do  
        #完整绝对路径
        path="$1$file"
        #判断是否是普通文件,如果是则进行编译链接
        if [ -f $path ];then
            exe=${file%.*}
            gcc -W -o $exe $path
            result=$?
            #获取gcc返回值,若为0则代表编译成功
            if [ $result -eq 0 ];then
                ./$exe
                rm $exe
            #否则输出错误,即非c程序
            else
                echo "$path is not a C program"
            fi
        #若为目录,输出信息,同时递归继续进行该函数	
        else
            echo $path is a Directory
            compile_run_C "$path/"
        fi
    done
    return 0
}
#如果未传入参数,则输出需要更多参数
if [ $# -eq 0 ]
then
    echo "This command requires more arguments"
    fileDiction=$(pwd)  
    echo "FIlE DIRECTION:$fileDiction"
fi
#获取全部传入参数的数组,针对每一个目录进行循环
for DIR in $@ 
do
    #如果传入非目录,则输出错误信息
    if [ ! -d "$DIR" ];then
        echo "Sorry,directory $DIR does not exist"
        exit 1
    #否则进行编译链接
    else
        compile_run_C $DIR 
    fi
done
        
    \end{lstlisting}
    \newpage
    \section{实验结果}
    \begin{figure}[h]
        \centering
        \includegraphics[width=15cm]{C:/PT_Latex/sources/lab1_1.png}
        \caption{ubuntu终端运行结果}
    \end{figure}

    \paragraph{可见:}
    \begin{itemize}
        \item 将/home/pt/c/和/home/pt/桌面/作为参数传入脚本,运行
        \item 成功运行4个c程序并输出
        \item 脚本检测到/home/pt/c/directory不是程序
        \item 进入/home/pt/c/directory后成功运行5.c并输出
        \item gcc编译报错,因为/home/pt/c/NoneC.txt不是c程序
        \item 脚本输出信息/home/pt/c/NoneC.txt is not a C program
        \item 进入第二个传入目录/home/pt/桌面/
        \item 成功运行outputHeart.c并输出
        \item gcc编译报错,因为/home/pt/桌面/text.txt不是c程序
        \item 脚本输出信息/home/pt/桌面/text.txt is not a C program
    \end{itemize}
    \section{问题分析}
    \subsection{出现的问题}
    \begin{enumerate}[i.]
        \item 对shell语言不熟悉导致各类语法错误,尤其是if else语句中忘记加入then、fi
        \item 传入参数时\$*和\$@的区别?
        \item 双引号什么时候必须加?
        \item 如何禁止gcc内部输出编译错误的信息?
    \end{enumerate}
    
    \subsection{解决办法}
    \begin{enumerate}[i.]
        \item 在菜鸟教程上自学shell语法并练习
        \item 目前认知是\$*将参数作为一个字符串传入,\$@将参数作为数组传入
        \item 暂时没有发现区别,似乎shell语言默认类型就是字符串?
        \item 尚不明确
    \end{enumerate}
\end{document}
