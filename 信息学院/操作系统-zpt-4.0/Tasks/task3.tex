\documentclass[UTF8]{ctexart}
\usepackage{dirtree}
\usepackage{listings}
\usepackage{xcolor}
\usepackage{graphicx}
\usepackage{enumerate}
\usepackage[a4paper]{geometry} 
\usepackage{amsmath}
\usepackage{mathtools}
\usepackage{diagbox}
\usepackage{multirow,makecell}

\title{Task 3}
\author{张配天-2018202180}
\date{}
\geometry{bottom=2cm}

\lstset{
 columns=fixed,       
 numbers=left,                                        % 在左侧显示行号
 numberstyle=\tiny\color{gray},                       % 设定行号格式
 basicstyle=\ttfamily,
 frame=none,                                          % 不显示背景边框
 backgroundcolor=\color[RGB]{245,245,244},            % 设定背景颜色
 keywordstyle=\color[RGB]{40,40,255},                 % 设定关键字颜色
 numberstyle=\footnotesize\color{darkgray},           
 commentstyle=\color{gray}\ttfamily,                % 设置代码注释的格式
 stringstyle=\rmfamily\slshape\color[RGB]{128,0,0},   % 设置字符串格式
 showstringspaces=false,
 breaklines=true,
 language=c++
}

\begin{document}
    \maketitle
    \section{5.6}
    \subsection{a}
    X=7,Y=28
    \subsection{b}
    \begin{table}[h]
        \centering
        \begin{tabular}{ccc}
            \hline
            operation&value of X&value of y\\
            LOAD R1,X&2&3\\
            LOAD R2,Y&2&3\\
            MUL R1,R2&2&3\\
            STORE X,R1&6&3\\
            INC R2&6&3\\
            STORE Y,R2&6&4\\
            LOAD R3,X&6&4\\
            INC R3&6&4\\
            LOAD R4,Y&6&4\\
            MUL R4,R3&6&4\\
            STORE X,R3&7&4\\
            STORE Y,R4&7&28\\
            \hline
        \end{tabular}
    \end{table}
    \subsection{c}
    可以用不同的调度使得X得到返回值6,Y返回值9
    \begin{table}[htb]
        \centering
        \begin{tabular}{ccc}
            \hline
            operation&value of X&value of Y\\
            LOAD R3,X&2&3\\
            INC R3&2&3\\
            LOAD R4,Y&2&3\\
            MUL R4,R3&2&3\\
            LOAD R1,X&2&3\\
            LOAD R2,Y&2&3\\
            MUL R1,R2&2&3\\
            STORE X,R3&2&3\\
            STORE X,R1&6&3\\
            INT R2&6&3\\
            STORE Y,R2&6&4\\
            STORE Y,R4&6&9\\
            \hline
        \end{tabular}
    \end{table}
    \newpage
    \section{5.26}
    \noindent 我们设立2个全局变量,用来记录精灵和驯鹿的个数:
    \begin{itemize}
        \item elves = 0,精灵数
        \item deers = 0,驯鹿数
    \end{itemize}
    \noindent 我们设立1个私用信号量,用来实现进程之间的同步:
    \begin{itemize}
        \item wakeup = 0,代表圣诞老人是否醒来
        \item solved = 0,代表精灵问题被解决
    \end{itemize}
    和4个公用信号量,用来实现进程之间的互斥:
    \begin{itemize}
        \item elfmutex = 1,用于精灵出问题的互斥锁
        \item elfproblem = 1,用于三个精灵出问题后的互斥
        \item deermutex = 1,用于驯鹿回北极的互斥锁
        
    \end{itemize}
    它们用来3个进程(不分顺序):
    \begin{itemize}
        \item \emph{圣诞老人进程}Santa\_Claus()
        \item \emph{小精灵进程}Elf()
        \item \emph{驯鹿进程}Reindeer()
    \end{itemize}
    这些进程中有函数:
    \begin{itemize}
        \item \emph{睡觉}sleep()
        \item \emph{起床}wakeup()
        \item \emph{组装雪橇}sleigh()
        \item \emph{解决问题}solve()
        \item \emph{做玩具}toy()
        \item \emph{遇到问题}problem()
        \item \emph{出精灵洞}leave()
        \item \emph{从热带回来}back()
        \item \emph{到达北极}arrive()
    \end{itemize}
    抽象代码如下:
    \begin{lstlisting}
    Santa\_Claus(){
        sleep();
        semWait(wakeup);
        if(deers == 9){
            sleigh()
            deers = 0;
        }
        else if(elves == 3){
            solve();
            semSignal(solved);
        }
    }
    Elf(){
        toy();
        if(problem()){
            semWait(elfproblem);
            semWait(elfmutex);
            elves++;
            if(elves == 3){
                semSignal(wakeup);
                leave()
            }
            else{
                semSignal(elfproblem)
            }
        }
        semWait(solved);
        semWait(elfmutex);
        elves--;
        if(elves == 0){
            semSignal(elfproblem);
        }
    }
    Reindeer(){
        semWait(deermutex);
        if(back()){
            deers++;
            if(deers == 9){
                semSignal(wakeup);
            }
        }
    }
    \end{lstlisting}
    \section{5.28}
    若有多个读者,有可能导致写入进程饿死。
\end{document}