\documentclass[UTF8]{ctexart}
\usepackage{dirtree}
\usepackage{listings}
\usepackage{xcolor}
\usepackage{graphicx}
\usepackage{enumerate}
\usepackage[a4paper]{geometry} 
\usepackage{amsmath,amsthm,mathtools}
\usepackage{mathtools}
\usepackage{diagbox}
\usepackage{multirow,makecell}
\usepackage{url}
\geometry{bottom=2cm}
\newcommand{\refe}[1]{Eq.\ref{#1}}
\newtheorem{theory}{Theory}[section]
\title{第十章作业}
\author{张配天-2018202180}
\begin{document}
    \maketitle
    \section*{11.1}
    设该进程所需的处理器运行时间为$T_p$,I/O设备的运行时间为$T_d$
    进程总相应时间为$T_e$
    无缓冲的情况下,$T_e = T_p + T_d$,加入单个缓冲区后,当$T_p = T_d$时,
    有$T_e = \frac{T_p+T_d}{2}$,这是所有情况中最短的时间。而最长时间的情况
    即$T_d$远远大(小)于$T_p$时。\par
    则$ \frac{T_p+T_d}{2} \leq T_e \leq T_p + T_d$
    得证。
    \section*{11.3}
    \begin{table}[htb]
        \centering
        \caption{FIFO}
        \begin{tabular}{cc}
            \hline
            Next truck accessed&移动的距离\\
            1045&205\\
            750&295\\
            932&182\\
            878&54\\
            1365&487\\
            1787&422\\
            1245&542\\
            664&581\\
            1678&1014\\
            1897&219\\
            \hline
        \end{tabular}
    \end{table}
    总距离:$205+295+181+54+487+422+542+581+1014+219=4001$
    \clearpage
    \begin{table}[htb]
        \centering
        \caption{SSTF}
        \begin{tabular}{cc}
            \hline
            Next truck accessed&移动的距离\\
            1245&5\\
            1365&120\\
            1678&313\\
            1787&109\\
            1897&110\\
            1045&848\\
            932&113\\
            878&54\\
            750&128\\
            664&86\\
            \hline
        \end{tabular}
    \end{table}
    总距离:$5+120+313+109+110+848+113+54+128+86=1886$
    
    \begin{table}[htb]
        \centering
        \caption{SCAN}
        \begin{tabular}{cc}
            \hline
            Next truck accessed&移动的距离\\
            1365&115\\
            1678&313\\
            1787&109\\
            1897&110\\
            1245&652\\
            1045&200\\
            932&113\\
            878&54\\
            750&128\\
            664&86\\
            \hline
        \end{tabular}
    \end{table}
    总距离:$115+313+109+110+652+200+113+54+128+86=1880$
    \clearpage
    \begin{table}[htb]
        \centering
        \caption{C-SCAN}
        \begin{tabular}{cc}
            \hline
            Next truck accessed&移动的距离\\
            1365&115\\
            1678&313\\
            1787&109\\
            1897&110\\
            664&1233\\
            750&86\\
            878&128\\
            932&54\\
            1045&113\\
            1245&200\\
            \hline
        \end{tabular}
    \end{table}
    总距离:$115+313+109+110+1233+86+128+54+113+200=2461$
\end{document}