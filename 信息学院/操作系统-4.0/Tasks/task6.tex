\documentclass[UTF8]{ctexart}
\usepackage{dirtree}
\usepackage{listings}
\usepackage{xcolor}
\usepackage{graphicx}
\usepackage{enumerate}
\usepackage[a4paper]{geometry} 
\usepackage{amsmath,amsthm,mathtools}
\usepackage{mathtools}
\usepackage{diagbox}
\usepackage{multirow,makecell}
\usepackage{url}
\geometry{bottom=2cm}
\newcommand{\refe}[1]{Eq.\ref{#1}}
\newtheorem{theory}{Theory}[section]
\title{task6}
\author{张配天-2018202180}
\begin{document}
    \maketitle
    \section*{8.1}
    \subsection*{a}
    设虚拟地址addr由$m+n$位组成,则操作系统会根据地址的前m位匹配对应的
    页框号,之后根据后n位计算偏移量,两者相连得到最终的物理地址。
    \subsection*{b}
    因为一页由2048个字节,则一页为$2^{11}=2kb$,后11位则为页内偏移量。
    \begin{enumerate}[i.]
        \item $6204 = 3^{11} + 60 = 11\ 00000111100$,因此对应页号为3,
        查页表,对应页框为6,则物理地址为$11000000111100 = 6*2^{11} + 60 = 12348$
        \item $3021 = 2^{11} + 973 = 10\ 01111001101$,因此对应页号为2,
        查页表,该页不在内存中,因此缺页中断。
        \item $9000 = 4*2^{11} + 808$,因此对应页号为4,
        查页表,对应页框为0,则物理地址为808.
    \end{enumerate}
    \clearpage
    \section*{8.4}

    \begin{table}
        \centering
        \caption{First-in-first-out}
        \label{ref1}
        \begin{tabular}{cccc}
        \hline
        页框1&页框2&页框3&是否置换\\
        a&&&\\
        a&b&&\\
        a&b&d&\\
        c&b&d&$\star$\\
        c&e&d&$\star$\\
        c&e&b&$\star$\\
        d&e&b&$\star$\\
        d&a&b&$\star$\\
        d&a&c&$\star$\\
        b&a&c&$\star$\\
        b&f&c&$\star$\\
        b&f&a&$\star$\\
        d&f&a&$\star$\\
        \hline
        \end{tabular}
        \begin{tabular}{cccc}
        \hline
        页框1&页框2&页框3&是否置换\\
        a&&\\
        a&b&\\
        a&b&d&$\star$\\
        c&b&d&$\star$\\
        e&b&d&$\star$\\
        a&b&d&$\star$\\
        a&b&c&$\star$\\
        a&f&c&$\star$\\
        d&f&c&\\
        \hline
        \end{tabular}
        \begin{tabular}{cccc}
        \hline
        页框1&页框2&页框3&是否置换\\
        a&&\\
        a&b&\\
        a&b&d&$\star$\\
        c&b&d&$\star$\\
        c&e&d&$\star$\\
        c&e&b&$\star$\\
        a&e&b&$\star$\\
        a&c&b&$\star$\\
        a&c&f&$\star$\\
        d&c&f&\\
        \hline
        \end{tabular}
    \end{table}
    
    \begin{table}
        \centering
        \caption{Optimal}
        \label{ref2}
        
    \end{table}
    
    \begin{table}
        \centering
        \caption{Least recently used}
        \label{ref3}
        \
    \end{table}
    综上,如\ref{ref1},\ref{ref2},\ref{ref3}所示,三种方式分别出现10次,6次,7次缺页中断,
    因此Optimal表现最好,FIFO表现最差。
    \section*{8.10}
    共$6G = 6*2^{30} \  bytes$内存空间,一页有$8*2^{10} \  bytes$,则内存中一共有
    $\frac{6*2^{30}}{8*2^{10}} = 3*2^{18}$页,又每页占$6 \  bytes$,则页表共需要
    $6*3*2^{18} = 9*2^{19} \  bytes = 4.5\ MB$
    \section*{8.14}  
    \subsection*{a}
    $8*2\ KB = 16\ KB$
    \subsection*{b}
    $4*16\ KB$
    \subsection*{c}
    将00021ABC转换为二进制得到0000\ 0000 \ 0000\ 00010\ 0001\ 1010\ 1011\ 1100
    其中段号占两个字节,页号占3个字节,页尺寸2KB,占11个字节。
    物理地址长度32位,物理地址空间最大为$2^{32} = 4\ GB$
\end{document}