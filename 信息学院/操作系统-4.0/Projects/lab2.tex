\documentclass[UTF8]{ctexart}
\usepackage{dirtree}
\usepackage{listings}
\usepackage{xcolor}
\usepackage{graphicx}
\usepackage{enumerate}
\usepackage[a4paper]{geometry} 
\usepackage{amsmath}
\usepackage{mathtools}
\usepackage{diagbox}
\usepackage{multirow,makecell}

\geometry{bottom=2cm}
\title{实验2 进程管理}
\author{张配天-2018202180}
\begin{document}
\maketitle
\tableofcontents
\thispagestyle{headings}
\newcommand{\reffig}[1]{\,\emph{图\ref{#1}}\,}
\newcommand{\includefig}[1]{\includegraphics[width=\textwidth]{#1}}

\lstset{
 columns=fixed,       
 numbers=left,                                        % 在左侧显示行号
 numberstyle=\tiny\color{gray},                       % 设定行号格式
 basicstyle=\ttfamily,
 frame=none,                                          % 不显示背景边框
 backgroundcolor=\color[RGB]{245,245,244},            % 设定背景颜色
 keywordstyle=\color[RGB]{40,40,255},                 % 设定关键字颜色
 numberstyle=\footnotesize\color{darkgray},           
 commentstyle=\color{gray}\ttfamily,                % 设置代码注释的格式
 stringstyle=\rmfamily\slshape\color[RGB]{128,0,0},   % 设置字符串格式
 showstringspaces=false,
 breaklines=true,
 language=c++
}

\section{实验目的}
编写一段c程序,实现子进程的创建、子进程的传递、捕捉信号、子进程的通信和加锁。

\section{实验方法}
因为对线程相关的指令不够熟悉,在学习作业要求附带的函数说明后又在网上查找相关资料,进行学习总结。\emph{下面针对每个题目给出思路}
\begin{enumerate}[I.]
    \item \textbf{创建两个子进程并输出信息(\emph{‘child i’})。}
    \begin{itemize}
        \item 记两个子进程分别为P1和P2
        \item 利用fork函数创建子进程,并利用switch语句操控主进程和子进程,使用全局变量pid1和pid2记录两个子进程的编号,方便在函数中访问
        \item 由于创建两个子进程,为了防止第一个子进程也创建二级子进程(\emph{孙子}),于是先fork一个子进程,在判断为父进程后再创建第二个
        \item 创建成功后直接在子进程中输出信息(\emph{‘child i’}),由于创建的时间有差异,因此会按顺序输出\emph{‘child 1’和‘child 2’}
    \end{itemize}
    \item \textbf{父进程捕捉键盘的中断信号,并使子进程输出信息(\emph{‘child process i is killed’})}
    \begin{itemize}
        \item 利用signal函数,在父进程中捕捉SIG\_INT并将其关联到父进程处理函数deposit\_ESC\_P,在该函数中子进程发送用户自定义信号16,
        \item 在P1中接受自定义信号16并将其关联到deposit\_ESC\_1,在该函数中输出信息(\emph{‘child process 1 is killed!’}),
        \item P2同理,区别在于不同的处理函数deposit\_ESC\_2
    \end{itemize}
    \item \textbf{利用管道实现子进程和父进程之间的通信,分别按顺序输出信息(\emph{‘child process i is sending a message’})}
    \begin{itemize}
        \item 利用pipe创建管道,利用close关闭特定端口,利用read和write分别实现读取和写入
        \item 父进程在fork前创建好管道,这样子进程才可以继承管道的权限,每个进程都掌握着管道的读取端和写入端。按照题目要求,父进程关闭写入端,
        两个子进程关闭读取端,并分别向管道中写入信息
    \end{itemize}
    \item \textbf{加锁后重新实现第三问功能}
    \begin{itemize}
        \item 利用lockf函数实现加锁和解锁
    \end{itemize}
\end{enumerate}

\emph{以上只是初步思路,在实现过程中遇到的数个问题及解决方案,均会在问题分析模块进行详细阐述}

\section{程序结构}
\dirtree{%
    .1 lab2.c.
    .2 pid1,pid2.
    .2 isToWait = 1 在自定义函数中修改,若为1则一直等待.
    .2 void waiting() 循环空转等待键盘指令.
    .2 deposit\_ESC\_1(int).
    .2 deposit\_ESC\_2(int).
    .2 deposit\_ESC\_P(int).
    .2 main().
    .3 P1(子进程).
    .3 P2(子进程).
}
\clearpage
\section{实验结果}
\noindent\textbf{在此处给出最终的结果,即实现了所有功能后的运行结果:}
\begin{figure}[h]
    \centering
    \caption{最终运行结果}
    \label{}
    \includegraphics[width=\textwidth]{C:/PT_Latex/sources/lock_sequent.png}
\end{figure}

\noindent\textbf{可见,实现了相应功能,且所有输出均按给定顺序}

\section{问题分析}
\textbf{针对每一问遇到的问题进行阐述:}
\begin{enumerate}[I.]
    \item 最开始不理解fork后程序运行规律,在实践中慢慢理解老师上课讲的fork就是把主进程变成两个同样
    程序,拥有相同的资源,并且执行同样的代码的程序(其中一父一子)。由此,值得注意的是,\textbf{在父进程中修改的
    信号关联,(比如将SIG\_INT关联到自定义函数上)不会在子进程中实现},这一点给我造成了很大的麻烦
    \item 这一问的关键在于signal函数捕捉信号,而难点在于输出顺序,下面分开说明: 
    \begin{itemize}
        \item 正如I所阐述,在这一问中要实现让子进程接受SIG\_INT并输出信息,绝不能仅仅在父进程中将SIG\_INT和
        和自定义函数关联起来,\textbf{在子进程中必须把SIG\_INT忽视},否则子进程会在键盘接收到\emph{ctrl+c}
        后按照默认将其判定为SIG\_INT,从而子进程陷入waiting函数的死循环之中,导致无法输出信息,正如\reffig{fig_1}所示
        
        \begin{figure}[htb]
            \centering        
            \caption{子进程未忽视SIG\_INT}
            \label{fig_1}
            \includegraphics[width=\textwidth,height=5cm]{C:/PT_Latex/sources/no_ignore.png}
        \end{figure}
        
        \item 同时,为了保证P1和P2的输出信息顺序,所以不能在deposit\_ESC\_P中直接分别向两个子进程发送信息,否则由于操作系统
        对于两个子进程调度的不确定性,会导致相反的输出顺序,如\reffig{fig_2}所示:
        \begin{figure}[htb]
            \centering
            \caption{同时向两个子进程发送自定义信号}
            \label{fig_2}
            \includefig{C:/PT_Latex/sources/no_lock_inverse.png}
        \end{figure}
        \\
        对此我们有两个解决方案:
        \begin{enumerate}
            \item 利用sleep函数,在父进程接收到键盘中断,向P1发送信号后休眠1ms再向P2发送信号,但不美观。
            \item 利用signal函数,让P1在接受主函数的信号并输出信息后再发信号给P2,这样就实现了顺序,且
            用全局变量保存pid1和pid2的优势得以体现。
        \end{enumerate}
    \end{itemize}

    \item 这一问关键在于两个进程向同一个管道的写入顺序,但在明确顺序之前,我还遇到了数个问题,下面分开说明:
    \begin{itemize}
        \item 无论是向管道写入还是向管道读入,必须要严格要求\textbf{读入和写入的字节数},否则会造成乱码,
        如\reffig{fig_3}\reffig{fig_4}所示
        \begin{figure}[htb]
            \centering
            \caption{读取的字符长度大于实际读取的字符长度}
            \label{fig_3}
            \includefig{C:/PT_Latex/sources/excess_read.png}
        \end{figure}
        \begin{figure}[htb]
            \centering
            \caption{写入的字符长度大于实际写入的字符长度}
            \label{fig_4}
            \includefig{C:/PT_Latex/sources/excess_write.png}
        \end{figure}
        \item 利用管道写入端的互斥性,用write函数直接写入字符串(一次性写入),保证两个进程写入互不干扰
        \item 利用deposit\_ESC\_1和deposit\_ESC\_2,调整P1向P2发信号代码的顺序(放在写入后),即可实现
        P1先写入,P2后写入的要求,若不然,则有可能造成相反的写入顺序
        \item 同时,\textbf{主函数必须使用wait等待P1,P2退出后再进行读取,否则会造成僵尸进程和类似的错误结果}
    \end{itemize}
    \item 如III所述,直接write字符串不会有两个进程的互相干扰,因此考虑把字符串的每一个字符循环输入到管道之中,
    对比如下:
    \begin{itemize}
        \item 不给循环写入的部分加锁,有可能导致两个进程争夺写入端,从而输出不连贯的字符串,如\reffig{fig_5}所示,\emph{因为字符串
        太短,发生这种事的概率极小}
        \begin{figure}[htb]
            \centering
            
            \caption{未加锁}
            \label{fig_5}
            \includefig{C:/PT_Latex/sources/no_lock_error_2.png}
        \end{figure}
        \item 加锁之后,进行数遍测试,未出现上述情况
    \end{itemize}
    
\end{enumerate}

\clearpage
\section{代码}
\begin{lstlisting}
    #include<stdio.h>
#include<pthread.h>
#include<signal.h>
#include<sys/types.h>
#include<unistd.h>
#include<string.h>
//解决wait的warning
#include <sys/wait.h>

int istoWait = 1;
int fd[2];
pid_t pid1,pid2;
char info_1[50] = "Child Process 1 is sending a message!\n";
char info_2[50] = "Child Process 2 is sending a message!\n";

void waiting(){
    while (istoWait == 1)
    {
        /* code */
    }
}

void deposit_ESC_1(int what){
    istoWait = 0;
    printf("Child Process 1 is Killed by Parent!\n");
    

    close(fd[0]);
        //向管道写入信息并关闭管道
        //用循环写入,这样方便对比加锁前后的输出情况
        lockf(1,1,0);
        for(int i = 0;i < strlen(info_1);i++){
            //printf("%c",info_1[i]);
            write(fd[1],info_1+i,1);
        }

        //write(fd[1],info_1,strlen(info_1));
        close(fd[1]);
        lockf(1,0,0);
        kill(pid2,17);
        exit(0);
} 
void deposit_ESC_2(int what){
    istoWait = 0;
    printf("Child Process 2 is Killed by Parent!\n");
    close(fd[0]);
            //写入管道并关闭通道
            //同上
            lockf(1,1,0);
            for(int i = 0;i < strlen(info_2);i++){
                //printf("%c",info_2[i]);
                write(fd[1],info_2+i,1);
            }
            //write(fd[1],info_2,strlen(info_1));
            
            close(fd[1]);
            lockf(1,0,0);
            exit(0);
} 
void deposit_ESC_P(int what){
    istoWait = 0;
     
    kill(pid1,16);

    //用sleep和信号都可以完成两个子进程的顺序
    //sleep(1);
    //kill(pid2,17);
} 

int main(){
    //int fd_2[2];
    if (pipe(fd) == -1)
    {
        printf("Error in creating pipes");
    }
    
    pid1 = fork();
    //int pid2 = fork();
    switch (pid1)
    {
        
    case 0/* constant-expression */:
        istoWait = 1;
        //忽视键盘中断,否则无法跳出循环
        signal(SIGINT,SIG_IGN);
        //关联自定义信号
        signal(16,deposit_ESC_1);
        printf("Child1\n");
        
        waiting();

        //关闭读入管道
        
        //lockf(1,0,0);
        break;
        
    case -1:
        printf("Error in creating process1\n");
        break;
    default:
        printf("Parent\n");
        //创建读入缓存区
        char buff[100];
        
        //if (pipe(fd_2) == -1)
        //{
        //    printf("Error in creating pipes");
        //}
        pid2 = fork();
        switch (pid2)
        {
        case 0:
            istoWait = 1;
            //关联信号
            signal(SIGINT,SIG_IGN);
            signal(17,deposit_ESC_2);
            printf("Child2\n");
            //等待键盘指令
            waiting();
            //关闭读入通道
            
            
            break;
        case -1:
            printf("Error in creating process2\n");
        default:
            
            //不做处理,因为和下面一样都是父进程
            break;
        }
        //关闭写入通道
        close(fd[1]);
        
        //关联信号,一定要把sigint忽视
        signal(SIGINT,deposit_ESC_P);
        //等待键盘指令
        waiting();
        //接受到ctrlC后发送信号
        
        //等待子进程结束
        //必须用循环,否则等待一个的话会造成僵尸进程和输出问题
        while(-1 != wait(0)){
            ;
        }
        //从同一个buff中读取信息
        read(fd[0],buff,sizeof(buff));
    
        //写入输出中
        write(STDOUT_FILENO,buff,strlen(info_1)+strlen(info_2));

        printf("Parent Process is killed!\n");
        exit(0);
        break;
    }
    return 0;
}
\end{lstlisting}
\end{document}